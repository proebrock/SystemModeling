\documentclass[a4paper,12pt]{article}

% Includes
\usepackage{epsfig}
\usepackage{amsfonts}    
\usepackage{amssymb}
\usepackage{color}
\usepackage{colortbl}
\usepackage{amsmath}
\usepackage[pdftex,bookmarks=true,pdfauthor={Philipp Roebrock},pdftitle={System Modeling}]{hyperref}
\usepackage[latin1]{inputenc}
\usepackage{lscape}
\usepackage{units}
\usepackage{rotating}
\usepackage{trsym}
\usepackage{minted}


\setcounter{secnumdepth}{3}
\setcounter{tocdepth}{3}

% Document's margins
\topmargin -1.5cm
\oddsidemargin -0.04cm
\evensidemargin -0.04cm
\textwidth 16.59cm
\textheight 25.5cm 
\parindent 0cm

%
% Include floating table
%
\newcommand{\TabStart}[1]{%
\begin{table}[htb]%
\begin{center}%
\begin{tabular}{#1}%
}
\newcommand{\TabEnd}[2]{%
\end{tabular}%
\end{center}%
\caption{#2\label{tab:#1}}%
\end{table}%
}
\newcommand{\TabRef}[1]{Table \ref{tab:#1}}

%
% Include floating plot
%
\newcommand{\Plot}[3]{%
\begin{figure}[htb]%
\begin{center}%
\resizebox{#2\textwidth}{!}{\input{#1.tex}}%
\end{center}%
\vspace{-5ex}%
\caption{#3\label{plot:#1}}%
\end{figure}%
}
\newcommand{\PlotRef}[1]{Figure \ref{plot:#1}}

%
% Include floating figure
%
\newcommand{\Fig}[4]{%
\begin{figure}[htb]%
\begin{center}%
\includegraphics[width=#2\textwidth]{#1.pdf}%
\end{center}%
\caption{#3\label{fig:#1}}%
\end{figure}%
}
\newcommand{\FigRef}[1]{Figure \ref{fig:#1}}

% Quotation marks
\newcommand{\Q}[1]{``{#1}''}

% Vectors and matrices
\renewcommand{\Vec}[1]{{\bf #1}}
\newcommand{\Mat}[1]{\mathbf{#1}}

\newcommand{\pfrac}[2]{\frac{#1}{#2}}

% Label and reference for chapters, sections, ...
\newcommand{\Chapter}[2]{\chapter{#2}\label{chapter:#1}}
\newcommand{\CRef}[1]{Chapter \ref{chapter:#1}}
\newcommand{\Section}[2]{\section{#2}\label{section:#1}}
\newcommand{\SRef}[1]{Section \ref{section:#1}}
\newcommand{\SubSection}[2]{\subsection{#2}\label{subsection:#1}}
\newcommand{\SSRef}[1]{Section \ref{subsection:#1}}
\newcommand{\SubSubSection}[2]{\subsubsection{#2}\label{subsubsection:#1}}
\newcommand{\SSSRef}[1]{Section \ref{subsubsection:#1}}
\newcommand{\ELabel}[1]{\label{equation:#1}}
\newcommand{\ERef}[1]{(\ref{equation:#1})}



\begin{document}

\vspace{1.5cm}
\begin{center}
{\Huge System Modeling}\\[3ex]
Philipp Roebrock\\[1ex]
\today\\
\end{center}
\vspace{1cm}



%%%%%%%%%%%%%%%%%%%%%%%%%%%%%%%%%%%%%%%%%%%%%%%%%%%%%%%%%%%%%%%%%%%%%%%%%%%%%%



\Section{theory}{Theory}

% System categories: Linear, time-invariant, continous-time

\SubSection{diffeq}{Differential equation}

Let the system be described by an $n$-th order {\em ordinary differential equation} ({\em ODE}):
\begin{eqnarray}
a_{n}\,\frac{d^{n}y(t)}{dt^{n}}&+&a_{n-1}\,\frac{d^{n-1}y(t)}{dt^{n-1}}+\ldots+a_{1}\,\frac{dy(t)}{dt}+a_{0}\,y(t)=\nonumber\\
b_{m}\,\frac{d^{m}u(t)}{dt^{m}}&+&b_{m-1}\,\frac{d^{m-1}u(t)}{dt^{m-1}}+\ldots+b_{1}\,\frac{du(t)}{dt}+b_{0}\,u(t)
\end{eqnarray}
Or in a more compact notation:
\begin{eqnarray}
\ELabel{ode}
a_{n}\,y^{(n)}(t)&+&a_{n-1}\,y^{(n-1)}(t)+\ldots+a_{1}\,\dot{y}(t)+a_{0}\,y(t)=\nonumber\\
b_{m}\,u^{(m)}(t)&+&b_{m-1}\,u^{(m-1)}(t)+\ldots+b_{1}\,\dot{u}(t)+b_{0}\,u(t)
\end{eqnarray}

\SubSection{xferfunc}{Transfer function}

The bilateral {\em Laplace transformation} of a function $f(t)$ is defined as
\begin{eqnarray}
F(s)=\mathcal{L}\left\{f(t)\right\}=\int\limits_{-\infty}^{\infty}f(t)\,e^{-s\,t}\,dt\qquad\forall\,t\geq0
\end{eqnarray}
and the back-transformation is given by
\begin{eqnarray}
f(t)=\mathcal{L}^{-1}\left\{F(s)\right\}=\frac{1}{2\pi j}\int\limits_{\sigma-j\infty}^{\sigma+j\infty}F(s)\,e^{s\,t}\,ds\qquad\forall\,t\geq0
\end{eqnarray}

After Laplace-transforming \ERef{ode} we get
\begin{eqnarray}
\ELabel{xfer}
a_{n}\,s^{n}\,y(s)&+&a_{n-1}\,s^{n-1}\,y(s)+\ldots+a_{1}\,s\,y(s)+a_{0}\,y(s)=\nonumber\\
b_{m}\,s^{m}\,u(s)&+&b_{m-1}\,s^{m-1}\,u(s)+\ldots+b_{1}\,s\,u(s)+b_{0}\,u(s)
\end{eqnarray}

The transfer function $G(s)$ is defined as
\begin{equation}
G(s):=\frac{y(s)}{u(s)}=\frac{b_{m}\,s^{m}+b_{m-1}\,s^{m-1}+\ldots+b_{1}\,s+b_{0}}{a_{n}\,s^{n}+a_{n-1}\,s^{n-1}+\ldots+a_{1}\,s+a_{0}}
\end{equation}

\SubSection{sspace}{State space representation}

For a given linear system with $p\in\mathbb{N}$ inputs, $q\in\mathbb{N}$
outputs and $n\in\mathbb{N}$ states, the system can be described by a set
of equations:
\begin{eqnarray}
\Vec{\dot{x}}(t)&=&\Mat{A}(t)\,\Vec{x}(t)+\Mat{B}(t)\,\Vec{u}(t)\\
\Vec{y}(t)&=&\Mat{C}(t)\,\Vec{x}(t)+\Mat{D}(t)\,\Vec{u}(t)
\end{eqnarray}
where $\Vec{u}(t)\in\mathbb{R}^p$ is the {\em input vector},
$\Vec{y}(t)\in\mathbb{R}^q$ the {\em output vector} and $\Vec{x}\in\mathbb{R}^n$
the {\em state vector}. The matrix $\Mat{A}$ with $\dim(\Mat{A})=n\times n$ is
the {\em state matrix}, $\Mat{B}$ with $\dim(\Mat{B})=n\times p$ is the
{\em input matrix}, $\Mat{C}$ with $\dim(\Mat{C})=q\times n$ is the
{\em output matrix}, $\Mat{D}$ with $\dim(\Mat{D})=q\times p$ is the
{\em feedthrough matrix}.\\

Typically we deal with {\em linear time invariant systems} ({\em LTI}) with 
$\Mat{A}$, $\Mat{B}$, $\Mat{C}$ and $\Mat{D}$ independent on $t$:
\begin{eqnarray}
\Vec{\dot{x}}(t)&=&\Mat{A}\,\Vec{x}(t)+\Mat{B}\,\Vec{u}(t)\\
\Vec{y}(t)&=&\Mat{C}\,\Vec{x}(t)+\Mat{D}\,\Vec{u}(t)
\end{eqnarray}

To see the block diagram representation of the state space equation, we refer
to \FigRef{statespace}.\\

\Fig{statespace}{0.8}{State space notation shown as block diagram}

For given differential equation like \ERef{ode} or transfer function like
\ERef{xfer} the state space 
\begin{equation}
\begin{array}{ll}
\Mat{A}=\left(\begin{array}{ccccc}
0 & 1 & 0 & \cdots & 0\\
\vdots & \ddots & \ddots & \ddots & \vdots\\
\vdots & & \ddots & \ddots & 0\\
0 & \cdots & \cdots & 0 & 1\\
-a_0 & -a_1 & \cdots & -a_{n-1} & -a_n
\end{array}\right)
&
\Mat{B}=\left(\begin{array}{c}
0\\
\vdots\\
\vdots\\
0\\
1
\end{array}\right)\\
\Mat{C}=\left(\begin{array}{ccccc}
b_0 & b_1 & \cdots & b_{n-1} & b_n
\end{array}\right)
&
\Mat{D}=\left(\begin{array}{c}
0
\end{array}\right)
\end{array}
\end{equation}

For a given state-space representation the transfer function can be calculated
by
\begin{equation}
G(s)=\Mat{C}\,\left(s\,\Mat{I}-\Mat{A}\right)^{-1}
\end{equation}

\SubSection{pulsestep}{Impulse and step response}

To determine the {\em impulse response} $h(t)$ of a system under the input of a
dirac impulse $\delta(t)$ we can use the inverse Laplace transformation of the
transfer function:
\begin{equation}
h(t)=\mathcal{L}^{-1}\left\{G(s)\right\}
\end{equation}
For the {\em step response} $a(t)$ of a system under the input of a unit step
function $H(t)$ we get
\begin{equation}
a(t)=\mathcal{L}^{-1}\left\{\frac{G(s)}{s}\right\}
\end{equation}
Impulse and step responses are important as basic properties of a system. On
the other way around it is possible to identify an unknown system by applying
a step to its input and observe the output.

\SubSection{simu}{Simulation}

\inputminted[tabsize=4]{python}{simulate.py}

%%%%%%%%%%%%%%%%%%%%%%%%%%%%%%%%%%%%%%%%%%%%%%%%%%%%%%%%%%%%%%%%%%%%%%%%%%%%%%



\Section{examples}{Examples}

\SubSection{oscillator}{Example A: Harmonic driven damped oscillator}

\SubSubSection{diffeqA}{Differential equation}

The differential equation of the driven damped harmonic oscillator is
\begin{equation}
\ddot{y}(t)+2\,\zeta\,\omega_0\,\dot{y}(t)+\omega_0^2\,y(t)=\beta\,u(t)
\end{equation}
with $\omega_0$ the angular frequency of the undamped oscillator in
$\unitfrac{1}{s}$, $\zeta$ the damping ratio (no unit) and $\beta$ a
conversion constant.\\

\paragraph{Electrical oscillator}

From the circuit diagram of a series RLC circuit (see \FigRef{elecoscil}) with
an electrical resistance $R$ in $\unit{\Omega}=\unitfrac{V}{A}$, an inductance
$L$ in $\unit{H}=\unitfrac{V\,s}{A}$ and a capacitance $C$ in
$\unit{F}=\unitfrac{A\,s}{V}$ we can derive:
\begin{eqnarray}
U_{In}(t)&=&L\,\frac{di(t)}{dt}+R\,i(t)+U_{Out}(t)\\
U_{Out}(t)&=&\frac{1}{C}\int i(t)dt\quad\Leftrightarrow\quad i(t)=C\,\frac{dU_{Out}(t)}{dt},\,\frac{di(t)}{dt}=C\,\frac{d^2U_{Out}(t)}{dt^2}\\
U_{In}(t)&=&L\,C\,\frac{d^2U_{Out}(t)}{dt^2}+R\,C\,\frac{dU_{Out}(t)}{dt}+U_{Out}\\
\frac{1}{L\,C}\,U_{In}(t)&=&\frac{d^2U_{Out}(t)}{dt^2}+\frac{R}{L}\,\frac{dU_{Out}(t)}{dt}+\frac{1}{L\,C}\,U_{Out}\\
\end{eqnarray}

\Fig{elecoscil}{0.5}{RLC circuit}

With $u=U_{In}$ being the driving voltage in $\unit{V}$ and $y=U_{Out}$
representing internal state and output voltage in $\unit{V}$, we get
\begin{equation}
\omega_0^2=\pfrac{1}{L\,C},\quad2\,\zeta\,\omega_0=\pfrac{R}{L},\quad\beta=\pfrac{1}{L\,C}
\end{equation}

\paragraph{Mechanical oscillator}

From the setup of a damped spring mass system (see \FigRef{mechoscil}) with a
mass $m$ in $\unit{kg}$, a spring constant $k$ in $\unitfrac{N}{m}$ and a
viscous damping coefficient $c$ in $\unitfrac{N\cdot s}{m}$ we can derive:
\begin{eqnarray}
F(t)&=&m\,\frac{d^2p(t)}{dt^2}+c\,\frac{dp(t)}{dt}+k\,p(t)\\
\frac{1}{m}\,F(t)&=&\frac{d^2p(t)}{dt^2}+\frac{c}{m}\,\frac{dp(t)}{dt}+\frac{k}{m}\,p(t)\\
\end{eqnarray}

\Fig{mechoscil}{0.5}{Damped spring-mass system}

With $u=F$ being the driving force in $\unit{N}$ and $y=p$ representing
internal state and position in $\unit{m}$, we get
\begin{equation}
\omega_0^2=\pfrac{k}{m},\quad2\,\zeta\,\omega_0=\pfrac{c}{m},\quad\beta=\pfrac{1}{m}
\end{equation}

% TODO: How does it look like when there is gravitation involved?

\SubSubSection{xferfuncA}{Transfer function}

\begin{eqnarray}
\ddot{y}(t)+2\,\zeta\,\omega_0\,\dot{y}(t)+\omega_0^2\,y(t)&=&\beta\,u(t)\\
&\TransformVert&\nonumber\\
s^2\,y(s)+2\,\zeta\,\omega_0\,s\,y(s)+\omega_0^2\,y(s)&=&\beta\,u(s)\\
G(s)&=&\frac{y(s)}{u(s)}=\frac{\beta}{s^2+2\,\zeta\,\omega_0\,s+\omega_0^2}
\end{eqnarray}

\SubSubSection{sspaceA}{State space representation}

We do a transformation by setting $x_1(t):=y(t)$ and $x_2(t):=\dot{y}(t)$. This
gives:
\begin{eqnarray}
	\dot{x_1}(t)&=&x_2(t)\\
	\dot{x_2}(t)&=&-2\,\zeta\,\omega_0\,x_2(t)-\omega_0^2\,x_1(t)+\beta\,u(t)
\end{eqnarray}
and written in matrix notation
\begin{equation}
\Vec{\dot{x}}(t)=\left(\begin{array}{cc}0 & 1\\-\omega_0^2 & -2\,\zeta\,\omega_0\end{array}\right)\,\Vec{x}(t)+\left(\begin{array}{c}0\\\beta\end{array}\right)\,u(t)
\end{equation}
Therefore we get with $p=1$, $q=1$ and $n=2$:
\begin{equation}
\begin{array}{ll}
	\Mat{A}=\left(\begin{array}{cc}
		0 & 1\\
		-\omega_0^2 & -2\,\zeta\,\omega_0
	\end{array}\right) &
	\Mat{B}=\left(\begin{array}{c}
		0\\
		\beta
	\end{array}\right)\\[4ex]
	\Mat{C}=\left(\begin{array}{cc}
		1 & 0
	\end{array}\right) &
	\Mat{D}=\left(\begin{array}{c}
		0
	\end{array}\right)\\
\end{array}
\end{equation}

\SubSubSection{pulsestepA}{Impulse and step response}

For $\zeta=0$ (undamped) we get
\begin{eqnarray}
G(s)&=&\frac{\beta}{s^2+\omega_0^2}\\
h(t)&=&\frac{\beta}{\omega_0}\,\sin(\omega_0\,t)\\
a(t)&=&\frac{\beta}{\omega_0^2}\left(1-\cos(\omega_0\,t)\right)
\end{eqnarray}
and for $\zeta=1$ (critically damped) we get
\begin{eqnarray}
G(s)&=&\frac{\beta}{s^2+2\,\omega_0\,s+\omega_0^2}\\
h(t)&=&\beta\,t\,e^{-\omega_0\,t}\\
a(t)&=&\frac{\beta}{\omega_0^2}\,\left(1-\left(1+t\,\omega_0\right)\,e^{-\omega_0\,t}\right)
\end{eqnarray}

\SubSubSection{simuA}{Simulation}

% analytical solution
% direct simulation, explaining ODE solver; python scipy simulation
% system identification

\end{document}

