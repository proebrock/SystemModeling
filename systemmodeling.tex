\documentclass[a4paper,12pt]{article}

% Includes
\usepackage{epsfig}
\usepackage{amsfonts}    
\usepackage{amssymb}
\usepackage{color}
\usepackage{colortbl}
\usepackage{amsmath}
\usepackage[pdftex,bookmarks=true,pdfauthor={Philipp Roebrock},pdftitle={System Modeling}]{hyperref}
\usepackage[latin1]{inputenc}
\usepackage{lscape}
\usepackage{units}
\usepackage{rotating}
\usepackage{trsym}
\usepackage{minted}
\usepackage{newfloat}


\setcounter{secnumdepth}{3}
\setcounter{tocdepth}{3}

% Document's margins
\topmargin -1.5cm
\oddsidemargin -0.04cm
\evensidemargin -0.04cm
\textwidth 16.59cm
\textheight 25.5cm 
\parindent 0cm

%
% Include floating table
%
\DeclareFloatingEnvironment[
	fileext=lor,
	listname=List of rows,
	name=Table,
	placement=htp,
	]{tabfloat}
% Temorary variables to allow newenvironment to access parameters from END part
\newcommand*{\TabLabel}{}
\newcommand*{\TabCaption}{}
\newenvironment{tab}[3]% {format}{label}{caption}
	% BEGIN part
	{\renewcommand*{\TabLabel}{#2}\renewcommand*{\TabCaption}{#3}\tabfloat\begin{center}\begin{tabular}{#1}}
	% END part
	{\end{tabular}\end{center}\caption{\TabCaption\label{code:\TabLabel}}\endcodefloat}
\newcommand{\TabRef}[1]{% {caption}
	Table \ref{tab:#1}%
	}

%
% Include floating figure
%
\newcommand{\Fig}[3]{% {filename of pdf}{zoom}{caption}
	\begin{figure}[htb]%
	\begin{center}%
	\includegraphics[width=#2\textwidth]{#1.pdf}%
	\end{center}%
	\caption{#3\label{fig:#1}}%
	\end{figure}%
	}
\newcommand{\FigRef}[1]{% {filename}
	Figure \ref{fig:#1}%
	}

%
% Include floating listing
%
\DeclareFloatingEnvironment[
	fileext=loc,
	listname=List of codes,
	name=Listing,
	placement=htp,
	]{codefloat}
% Temorary variables to allow newenvironment to access parameters from END part
\newcommand*{\CodeLabel}{}
\newcommand*{\CodeCaption}{}
\newenvironment{code}[4]% {language}{label}{caption}{options}
	% BEGIN part
	{\renewcommand*{\CodeLabel}{#2}\renewcommand*{\CodeCaption}{#3}\codefloat\hrulefill\minted[#4]{#1}}
	% END part
	{\endminted\caption{\CodeCaption\label{code:\CodeLabel}}\hrulefill\endcodefloat}
\newcommand{\Code}[3]{% {filename}{language}{caption}
	\codefloat\hrulefill%
	\inputminted[tabsize=4]{#2}{#1}%
	\caption{#3\label{code:#1}}\hrulefill\endcodefloat%
	}
\newcommand{\CodeRef}[1]{% {caption or filename}
	Listing \ref{code:#1}%
	}

% Quotation marks
\newcommand{\Q}[1]{``{#1}''}

% Vectors and matrices
\renewcommand{\Vec}[1]{{\bf #1}}
\newcommand{\Mat}[1]{\mathbf{#1}}

\newcommand{\pfrac}[2]{\frac{#1}{#2}}

% Label and reference for chapters, sections, ...
\newcommand{\Chapter}[2]{\chapter{#2}\label{chapter:#1}}
\newcommand{\CRef}[1]{Chapter \ref{chapter:#1}}
\newcommand{\Section}[2]{\section{#2}\label{section:#1}}
\newcommand{\SRef}[1]{Section \ref{section:#1}}
\newcommand{\SubSection}[2]{\subsection{#2}\label{subsection:#1}}
\newcommand{\SSRef}[1]{Section \ref{subsection:#1}}
\newcommand{\SubSubSection}[2]{\subsubsection{#2}\label{subsubsection:#1}}
\newcommand{\SSSRef}[1]{Section \ref{subsubsection:#1}}
\newcommand{\ELabel}[1]{\label{equation:#1}}
\newcommand{\ERef}[1]{(\ref{equation:#1})}



\begin{document}

\vspace{1.5cm}
\begin{center}
{\Huge System Modeling}\\[3ex]
Philipp Roebrock\\[1ex]
\today\\
\end{center}
\vspace{1cm}



\Section{diffeq}{Differential equation}

A dynamical system can be described by an $n$-th order {\em ordinary differential
equation} ({\em ODE})
\begin{eqnarray}
a_{n}\,\frac{d^{n}y(t)}{dt^{n}}&+&a_{n-1}\,\frac{d^{n-1}y(t)}{dt^{n-1}}+\ldots+a_{1}\,\frac{dy(t)}{dt}+a_{0}\,y(t)=\nonumber\\
b_{m}\,\frac{d^{m}u(t)}{dt^{m}}&+&b_{m-1}\,\frac{d^{m-1}u(t)}{dt^{m-1}}+\ldots+b_{1}\,\frac{du(t)}{dt}+b_{0}\,u(t)
\end{eqnarray}
or the same written in a more compact notation
\begin{eqnarray}
\ELabel{ode}
a_{n}\,y^{(n)}(t)&+&a_{n-1}\,y^{(n-1)}(t)+\ldots+a_{1}\,y'(t)+a_{0}\,y(t)=\nonumber\\
b_{m}\,u^{(m)}(t)&+&b_{m-1}\,u^{(m-1)}(t)+\ldots+b_{1}\,u'(t)+b_{0}\,u(t)
\end{eqnarray}

\paragraph{Example}

An example for a dynamic system is the {\em driven damped harmonic oscillator}
which has this differential equation
\begin{equation}
\ELabel{oscillatorode}
\ddot{y}(t)+2\,\zeta\,\omega_0\,\dot{y}(t)+\omega_0^2\,y(t)=\beta\,u(t)
\end{equation}
with $\omega_0$ the angular frequency of the undamped oscillator in
$\unitfrac{1}{s}$, $\zeta$ the damping ratio (no unit) and $\beta$ a
conversion constant.\\

{\em Electrical oscillator}: From the circuit diagram of a series RLC circuit
(see \FigRef{elecoscil}) with an electrical resistance $R$ in
$\unit{\Omega}=\unitfrac{V}{A}$, an inductance $L$ in $\unit{H}=\unitfrac{V\,s}{A}$
and a capacitance $C$ in $\unit{F}=\unitfrac{A\,s}{V}$ we can derive:
\begin{eqnarray}
U_{In}(t)&=&L\,\frac{di(t)}{dt}+R\,i(t)+U_{Out}(t)\\
U_{Out}(t)&=&\frac{1}{C}\int i(t)dt\quad\Leftrightarrow\quad i(t)=C\,\frac{dU_{Out}(t)}{dt},\,\frac{di(t)}{dt}=C\,\frac{d^2U_{Out}(t)}{dt^2}\\
U_{In}(t)&=&L\,C\,\frac{d^2U_{Out}(t)}{dt^2}+R\,C\,\frac{dU_{Out}(t)}{dt}+U_{Out}\\
\frac{1}{L\,C}\,U_{In}(t)&=&\frac{d^2U_{Out}(t)}{dt^2}+\frac{R}{L}\,\frac{dU_{Out}(t)}{dt}+\frac{1}{L\,C}\,U_{Out}\\
\end{eqnarray}

\Fig{elecoscil}{0.5}{RLC circuit}

With $u=U_{In}$ being the driving voltage in $\unit{V}$ and $y=U_{Out}$
representing internal state and output voltage in $\unit{V}$, we get
\begin{equation}
\omega_0^2=\pfrac{1}{L\,C},\quad2\,\zeta\,\omega_0=\pfrac{R}{L},\quad\beta=\pfrac{1}{L\,C}
\end{equation}

{\em Mechanical oscillator}: From the setup of a damped spring mass system
(see \FigRef{mechoscil}) with a mass $m$ in $\unit{kg}$, a spring constant $k$
in $\unitfrac{N}{m}$ and a viscous damping coefficient $c$ in
$\unitfrac{N\cdot s}{m}$ we can derive:
\begin{eqnarray}
F(t)&=&m\,\frac{d^2p(t)}{dt^2}+c\,\frac{dp(t)}{dt}+k\,p(t)\\
\frac{1}{m}\,F(t)&=&\frac{d^2p(t)}{dt^2}+\frac{c}{m}\,\frac{dp(t)}{dt}+\frac{k}{m}\,p(t)\\
\end{eqnarray}

\Fig{mechoscil}{0.5}{Damped spring-mass system}

With $u=F$ being the driving force in $\unit{N}$ and $y=p$ representing
internal state and position in $\unit{m}$, we get
\begin{equation}
\omega_0^2=\pfrac{k}{m},\quad2\,\zeta\,\omega_0=\pfrac{c}{m},\quad\beta=\pfrac{1}{m}
\end{equation}

% TODO: How does it look like when there is gravitation involved?

\Section{xferfunc}{Transfer function}

Basis of the {\em transfer function} is the bilateral {\em Laplace
transformation} of a function $f(t)$ which is defined as
\begin{eqnarray}
F(s)=\mathcal{L}\left\{f(t)\right\}=\int\limits_{-\infty}^{\infty}f(t)\,e^{-s\,t}\,dt\qquad\forall\,t\geq0
\end{eqnarray}
and the back-transformation is given by
\begin{eqnarray}
f(t)=\mathcal{L}^{-1}\left\{F(s)\right\}=\frac{1}{2\pi j}\int\limits_{\sigma-j\infty}^{\sigma+j\infty}F(s)\,e^{s\,t}\,ds\qquad\forall\,t\geq0
\end{eqnarray}

After Laplace-transforming \ERef{ode} we get
\begin{eqnarray}
\ELabel{xfer}
a_{n}\,s^{n}\,y(s)&+&a_{n-1}\,s^{n-1}\,y(s)+\ldots+a_{1}\,s\,y(s)+a_{0}\,y(s)=\nonumber\\
b_{m}\,s^{m}\,u(s)&+&b_{m-1}\,s^{m-1}\,u(s)+\ldots+b_{1}\,s\,u(s)+b_{0}\,u(s)
\end{eqnarray}

The transfer function $G(s)$ is defined as
\begin{equation}
G(s):=\frac{y(s)}{u(s)}=\frac{b_{m}\,s^{m}+b_{m-1}\,s^{m-1}+\ldots+b_{1}\,s+b_{0}}{a_{n}\,s^{n}+a_{n-1}\,s^{n-1}+\ldots+a_{1}\,s+a_{0}}
\end{equation}
If $m\leq n$, the transfer function is called {\em proper}.\\

\paragraph{Example}

To determine the transfer function of the harmonic oscillator we
Laplace-transform the differential equation \ERef{oscillatorode}
\begin{eqnarray}
\ddot{y}(t)+2\,\zeta\,\omega_0\,\dot{y}(t)+\omega_0^2\,y(t)&=&\beta\,u(t)\\
&\TransformVert&\nonumber\\
s^2\,y(s)+2\,\zeta\,\omega_0\,s\,y(s)+\omega_0^2\,y(s)&=&\beta\,u(s)\\
G(s)&=&\frac{y(s)}{u(s)}=\frac{\beta}{s^2+2\,\zeta\,\omega_0\,s+\omega_0^2}
\end{eqnarray}

To define the mechanical oscillator in Python with the transfer function
\begin{equation}
G(s)=\frac{1}{m\,s^2+c\,s+k}
\end{equation}
we just write
\begin{minted}{python}
sys = signal.lti([ 1 ], [ m, c, k ])
\end{minted}

\Section{sspace}{State space representation}

For a given linear system with $p\in\mathbb{N}$ inputs, $q\in\mathbb{N}$
outputs and $n\in\mathbb{N}$ states, the system can be described by a set
of equations:
\begin{eqnarray}
\Vec{\dot{x}}(t)&=&\Mat{A}\,\Vec{x}(t)+\Mat{B}\,\Vec{u}(t)\\
\Vec{y}(t)&=&\Mat{C}\,\Vec{x}(t)+\Mat{D}\,\Vec{u}(t)
\end{eqnarray}
where $\Vec{u}(t)\in\mathbb{R}^p$ is the {\em input vector},
$\Vec{y}(t)\in\mathbb{R}^q$ the {\em output vector} and $\Vec{x}\in\mathbb{R}^n$
the {\em state vector}. The matrix $\Mat{A}$ with $\dim(\Mat{A})=n\times n$ is
the {\em state matrix}, $\Mat{B}$ with $\dim(\Mat{B})=n\times p$ is the
{\em input matrix}, $\Mat{C}$ with $\dim(\Mat{C})=q\times n$ is the
{\em output matrix}, $\Mat{D}$ with $\dim(\Mat{D})=q\times p$ is the
{\em feedthrough matrix}.\\

To see the block diagram representation of the state space equation, we refer
to \FigRef{statespace}.\\

\Fig{statespace}{0.8}{State space notation shown as block diagram}

For given transfer function like \ERef{xfer} that is proper, the state space
representation can be expressed by so called {\em canonical representations}.
The {\em controllable canonical form} is given by
\begin{equation}
\begin{array}{ll}
\Mat{A}=\left(\begin{array}{ccccc}
0 & 1 & 0 & \cdots & 0\\
\vdots & \ddots & \ddots & \ddots & \vdots\\
\vdots & & \ddots & \ddots & 0\\
0 & \cdots & \cdots & 0 & 1\\
-\frac{a_0}{a_n} & -\frac{a_1}{a_n} & \cdots & -\frac{a_{n-2}}{a_n} & -\frac{a_{n-1}}{a_n}
\end{array}\right)
&
\Mat{B}=\left(\begin{array}{c}
0\\
\vdots\\
\vdots\\
0\\
1
\end{array}\right)\\
\Mat{C}=\left(\begin{array}{ccccc}
\frac{a_n\,b_0-a_0\,b_n}{a_n^2} & \frac{a_n\,b_1-a_1\,b_n}{a_n^2} & \cdots & \frac{a_n\,b_{n-1}-a_{n-1}\,b_n}{a_n^2}
\end{array}\right)
&
\Mat{D}=\left(\begin{array}{c}
\frac{b_n}{a_n}
\end{array}\right)
\end{array}
\end{equation}

For a given state-space representation the transfer function can be calculated
by
\begin{equation}
G(s)=\Mat{C}\,\left(s\,\Mat{I}-\Mat{A}\right)^{-1}\,\Mat{B}+\Mat{D}
\end{equation}

\paragraph{Example}

We tranform the differential equation of the harmonic oscillator
\ERef{oscillatorode} by setting $x_1(t):=y(t)$ and $x_2(t):=\dot{y}(t)$.
This gives:
\begin{eqnarray}
	\dot{x_1}(t)&=&x_2(t)\\
	\dot{x_2}(t)&=&-2\,\zeta\,\omega_0\,x_2(t)-\omega_0^2\,x_1(t)+\beta\,u(t)
\end{eqnarray}
and written in matrix notation
\begin{equation}
\Vec{\dot{x}}(t)=\left(\begin{array}{cc}0 & 1\\-\omega_0^2 & -2\,\zeta\,\omega_0\end{array}\right)\,\Vec{x}(t)+\left(\begin{array}{c}0\\\beta\end{array}\right)\,u(t)
\end{equation}
Therefore we get with $p=1$, $q=1$ and $n=2$:
\begin{equation}
\begin{array}{ll}
	\Mat{A}=\left(\begin{array}{cc}
		0 & 1\\
		-\omega_0^2 & -2\,\zeta\,\omega_0
	\end{array}\right) &
	\Mat{B}=\left(\begin{array}{c}
		0\\
		\beta
	\end{array}\right)\\[4ex]
	\Mat{C}=\left(\begin{array}{cc}
		1 & 0
	\end{array}\right) &
	\Mat{D}=\left(\begin{array}{c}
		0
	\end{array}\right)\\
\end{array}
\end{equation}

To define a system in Python given in state space representation we just
\begin{minted}{python}
sys = signal.lti(A, B, C, D)
\end{minted}

\Section{pulsestep}{Impulse and step response}

To determine the {\em impulse response} $h(t)$ of a system under the input of a
dirac impulse $\delta(t)$ we can use the inverse Laplace transformation of the
transfer function:
\begin{equation}
h(t)=\mathcal{L}^{-1}\left\{G(s)\right\}
\end{equation}
For the {\em step response} $a(t)$ of a system under the input of a unit step
function $H(t)$ we get
\begin{equation}
a(t)=\mathcal{L}^{-1}\left\{\frac{G(s)}{s}\right\}
\end{equation}
Impulse and step responses are important as basic properties of a system
telling you much about its behavior. It is possible to identify an unknown
system by applying a step to its input and observe the output.\\

\paragraph{Example}

For the harmonic oscillator and $\zeta=0$ (undamped) we get
\begin{eqnarray}
G(s)&=&\frac{\beta}{s^2+\omega_0^2}\\
h(t)&=&\frac{\beta}{\omega_0}\,\sin(\omega_0\,t)\\
a(t)&=&\frac{\beta}{\omega_0^2}\left(1-\cos(\omega_0\,t)\right)
\end{eqnarray}
and for $\zeta=1$ (critically damped) we get
\begin{eqnarray}
G(s)&=&\frac{\beta}{s^2+2\,\omega_0\,s+\omega_0^2}\\
h(t)&=&\beta\,t\,e^{-\omega_0\,t}\\
a(t)&=&\frac{\beta}{\omega_0^2}\,\left(1-\left(1+t\,\omega_0\right)\,e^{-\omega_0\,t}\right)
\end{eqnarray}

To numerically simulate the impulse response of the harmonic oscillator, see
\CodeRef{impulse}. The result of the simulation can be seen in \FigRef{impulse}.
The impulse initates an oscillation of the system around zero with a cycle time
\begin{eqnarray}
T=\frac{2\,\pi}{\omega_0}=\frac{2\,\pi}{\sqrt{\frac{k}{m}}}=\frac{2\,\pi}{\sqrt{\frac{0.8}{1.0}}}\approx 7\unit{s}.
\end{eqnarray}
The damping of the oscillator makes the oscillation die down after some time.\\

\begin{code}{python}{impulse}{Simulate the impulse response of a system}{linenos}
# Define system
m = 1.0 # Mass [kg]
k = 0.8 # Spring constant [N/m]
c = 0.3 # Damping constant [Ns/m]
sys = signal.lti([ 1 ], [ m, c, k ])
# Time range
n = 100+1
t = np.linspace(0, 40, num=n)
# Simulate
T, yout = signal.impulse(sys, T=t)
# Plot result
plt.plot(T, yout)
plt.grid()
plt.xlabel('Time (s)')
plt.ylabel('Output (m)')
plt.show()
\end{code}

\Fig{impulse}{0.7}{Impulse response simulation result}

To simulate the step response use \CodeRef{impulse} and replace line 10 with
\begin{minted}{python}
T, yout = signal.step(sys, T=t).
\end{minted}
The result is shown in \FigRef{step}. Again the step input initiates an
oscillation with the same cycle time as for the impulse response. After the
oscillation died down, the new static position of the oscillator is
\begin{eqnarray}
x=\frac{F}{k}=\frac{1}{0.8}=1.25\unit{m}.
\end{eqnarray}

\Fig{step}{0.7}{Step response simulation result}

\Section{simulate}{Simulation}



\begin{minted}{python}
# Define system
m = 1.0 # Mass [kg]
k = 0.8 # Spring constant [N/m]
c = 0.0 # Damping constant [Ns/m]
sys = signal.lti([ 1 ], [ m, c, k ])
# Time range
n = 500+1
t = np.linspace(0, 40, num=n)
# Input
U = np.sin(np.sqrt(k/m)*t)
# Simulate
T, yout, xout = signal.lsim(sys, U, t, X0=np.array([ 0, 0 ]))
# Plot result
plt.plot(T, U, label='Input $U(t)$')
plt.plot(T, yout, label='Output $y(t)$')
plt.grid()
plt.xlabel('Time (s)')
plt.legend(loc='upper left')
plt.show()
\end{minted}

\Fig{simulate_lsim}{0.7}{Simulation result}

% TODO: analytical solution
% TODO: direct simulation, explaining ODE solver; python scipy simulation
% TODO: system identification
% TODO: Lotka-Volterra, Lorenz, gravitational 3-body-problem
% TODO: Vector fields

\end{document}

